\documentclass{article}
\usepackage{multicol}
\usepackage{fancyhdr}


\title{日本語で TeX Live on Fedora のサンプル出力}
\author{佐藤 暁}


% custom layout rules
\setlength{\oddsidemargin}{-12mm}
\setlength{\textwidth}{520pt}
\setlength{\textheight}{650pt}
\setlength{\topmargin}{0pt}
\setlength{\voffset}{-10pt}
%\setlength{\baselineskip}{10pt}
%\setlength{\headheight}{45pt}


% custom header
\fancyhead{}
\fancyfoot{}
\fancyhead[C]{日本語で TeX Live on Fedora のサンプル PDF 出力}
\fancyfoot[C]{フッターのテスト}
\renewcommand{\headrulewidth}{0pt}
\renewcommand{\footrulewidth}{0pt}


\begin{document}
%\maketitle


\begin{center}
{\textbf{\large 日本語で TeX Live on Fedora のサンプル PDF 出力}}
\end{center}


こんにちは、TeX Live! Tex はすっかり書き方を忘れたけれどもこんな感じ? とりあえず、段組のテスト。


\begin{multicols}{2}

\section{あたらしい憲法のはなし}

以下は青空文庫の『あたらしい憲法のはなし』 からの引用:

%\begin{quote}
みなさん、あたらしい憲法ができました。そうして昭和二十二年五月三日から、私たち日本國民は、この憲法を守ってゆくことになりました。このあたらしい憲法をこしらえるために、たくさんの人々が、たいへん苦心をなさいました。ところでみなさんは、憲法というものはどんなものかごぞんじですか。じぶんの身にかゝわりのないことのようにおもっている人はないでしょうか。もしそうならば、それは大きなまちがいです。


國の仕事は、一日も休むことはできません。また、國を治めてゆく仕事のやりかたは、はっきりときめておかなければなりません。そのためには、いろ/\規則がいるのです。この規則はたくさんありますが、そのうちで、いちばん大事な規則が憲法です。


國をどういうふうに治め、國の仕事をどういうふうにやってゆくかということをきめた、いちばん根本になっている規則が憲法です。もしみなさんの家の柱がなくなったとしたらどうでしょう。家はたちまちたおれてしまうでしょう。いま國を家にたとえると、ちょうど柱にあたるものが憲法です。もし憲法がなければ、國の中におゝぜいの人がいても、どうして國を治めてゆくかということがわかりません。それでどこの國でも、憲法をいちばん大事な規則として、これをたいせつに守ってゆくのです。國でいちばん大事な規則は、いいかえれば、いちばん高い位にある規則ですから、これを國の「最高法規」というのです。


ところがこの憲法には、いまおはなししたように、國の仕事のやりかたのほかに、もう一つ大事なことが書いてあるのです。それは國民の権利のことです。この権利のことは、あとでくわしくおはなししますから、こゝではたゞ、なぜそれが、國の仕事のやりかたをきめた規則と同じように大事であるか、ということだけをおはなししておきましょう。


みなさんは日本國民のうちのひとりです。國民のひとり/\が、かしこくなり、強くならなければ、國民ぜんたいがかしこく、また、強くなれません。國の力のもとは、ひとり/\の國民にあります。そこで國は、この國民のひとり/\の力をはっきりとみとめて、しっかりと守ってゆくのです。そのために、國民のひとり/\に、いろ/\大事な権利があることを、憲法できめているのです。この國民の大事な権利のことを「基本的人権」というのです。これも憲法の中に書いてあるのです。


そこでもういちど、憲法とはどういうものであるかということを申しておきます。憲法とは、國でいちばん大事な規則、すなわち「最高法規」というもので、その中には、だいたい二つのことが記されています。その一つは、國の治めかた、國の仕事のやりかたをきめた規則です。もう一つは、國民のいちばん大事な権利、すなわち「基本的人権」をきめた規則です。このほかにまた憲法は、その必要により、いろ/\のことをきめることがあります。こんどの憲法にも、あとでおはなしするように、これからは戰爭をけっしてしないという、たいせつなことがきめられています。


これまであった憲法は、明治二十二年にできたもので、これは明治天皇がおつくりになって、國民にあたえられたものです。しかし、こんどのあたらしい憲法は、日本國民がじぶんでつくったもので、日本國民ぜんたいの意見で、自由につくられたものであります。この國民ぜんたいの意見を知るために、昭和二十一年四月十日に総選挙が行われ、あたらしい國民の代表がえらばれて、その人々がこの憲法をつくったのです。それで、あたらしい憲法は、國民ぜんたいでつくったということになるのです。


みなさんも日本國民のひとりです。そうすれば、この憲法は、みなさんのつくったものです。みなさんは、じぶんでつくったものを、大事になさるでしょう。こんどの憲法は、みなさんをふくめた國民ぜんたいのつくったものであり、國でいちばん大事な規則であるとするならば、みなさんは、國民のひとりとして、しっかりとこの憲法を守ってゆかなければなりません。そのためには、まずこの憲法に、どういうことが書いてあるかを、はっきりと知らなければなりません。


みなさんが、何かゲームのために規則のようなものをきめるときに、みんないっしょに書いてしまっては、わかりにくい[#「わかりにくい」は底本では「わかりくい」]でしょう。國の規則もそれと同じで、一つ/\事柄にしたがって分けて書き、それに番号をつけて、第何條、第何條というように順々に記します。こんどの憲法は、第一條から第百三條まであります。そうしてそのほかに、前書が、いちばんはじめにつけてあります。これを「前文」といいます。


この前文には、だれがこの憲法をつくったかということや、どんな考えでこの憲法の規則ができているかということなどが記されています。この前文というものは、二つのはたらきをするのです。その一つは、みなさんが憲法をよんで、その意味を知ろうとするときに、手びきになることです。つまりこんどの憲法は、この前文に記されたような考えからできたものですから、前文にある考えと、ちがったふうに考えてはならないということです。もう一つのはたらきは、これからさき、この憲法をかえるときに、この前文に記された考え方と、ちがうようなかえかたをしてはならないということです。
%\end{quote}

\end{multicols}


\section{GNU Manifesto}
以下は GNU Manifesto 1993年改訂の日本語訳
(http://www.gnu.org/japan/manifesto-1993j-plain.html) からの引用:


\begin{quote}
長い目で見た場合には、プログラムをフリーにすることは、欠乏の無い世界へ
の第一歩であり、そこでは誰も生計を立てるためだけにあくせく働く必要はな
いだろう。人々は、週に10時間の課せられた仕事、例えば、法律の制定や、家
族との相談、ロボットの修理、小惑星の試掘といった必要な仕事をこなしたあ
とは、プログラミングといった楽しめる活動に自由に専念することになるだろ
う。もはやプログラミングで生計を立てる必要はなくなる。


我々は既に、社会全体が実質的生産のためにしなければならない作業量を大幅
に減らしてきたが、そのうちのほんのわずかが労働者の娯楽に変わっただけで
ある。というのは、生産活動に伴い多くの非生産活動が必要とされるからであ
る。その主な原因は、官僚主義と競争に対する差の無い骨折りである。フリー・
ソフトウェアは、ソフトウェア生産の分野でこれらの乱費流出を大幅に減らす
だろう。生産における技術的利得が我々にとっての労働の軽減になるよう、我々
はこれを行なっていかなければならないのである。
\end{quote}


\section{GNU GPL}
以下は GNU GPL の日本語訳
(http://www.opensource.jp/gpl/gpl.ja.html) からの引用:


\begin{quote}
ソフトウェア向けライセンスの大半は、あなたがそのソフトウェアを共有したり変更したりする自由を奪うように設計されています。対照的に、GNU 一般公衆利用許諾契約書は、あなたがフリーソフトウェアを共有したり変更したりする自由を保証する--すなわち、ソフトウェアがそのユーザすべてにとってフリーであることを保証することを目的としています。この一般公衆利用許諾契約書はフリーソフトウェア財団のソフトウェアのほとんどに適用されており、また GNU GPLを適用すると決めたフリーソフトウェア財団以外の作者によるプログラムにも適用されています(いくつかのフリーソフトウェア財団のソフトウェアには、GNU GPLではなくGNU ライブラリ一般公衆利用許諾契約書が適用されています)。あなたもまた、ご自分のプログラムにGNU GPLを適用することが可能です。


私たちがフリーソフトウェアと言うとき、それは利用の自由について言及しているのであって、価格は問題にしていません。私たちの一般公衆利用許諾契約書は、あなたがフリーソフトウェアの複製物を頒布する自由を保証するよう設計されています(希望に応じてその種のサービスに手数料を課す自由も保証されます)。また、あなたがソースコードを受け取るか、あるいは望めばそれを入手することが可能であるということ、あなたがソフトウェアを変更し、その一部を新たなフリーのプログラムで利用できるということ、そして、以上で述べたようなことができるということがあなたに知らされるということも保証されます。


あなたの権利を守るため、私たちは誰かがあなたの有するこれらの権利を否定することや、これらの権利を放棄するよう要求することを禁止するという制限を加える必要があります。よって、あなたがソフトウェアの複製物を頒布したりそれを変更したりする場合には、そういった制限のためにあなたにある種の責任が発生することになります。


例えば、あなたがフリーなプログラムの複製物を頒布する場合、有料か無料に関わらず、あなたは自分が有する権利を全て受領者に与えなければなりません。また、あなたは彼らもソースコードを受け取るか手に入れることができるよう保証しなければなりません。そして、あなたは彼らに対して以下で述べる条件を示し、彼らに自らの持つ権利について知らしめるようにしなければなりません。


私たちはあなたの権利を二段階の手順を踏んで保護します。(1) まずソフトウェアに対して著作権を主張し、そして (2) あなたに対して、ソフトウェアの複製や頒布または改変についての法的な許可を与えるこの契約書を提示します。


また、各作者や私たちを保護するため、私たちはこのフリーソフトウェアには何の保証も無いということを誰もが確実に理解するようにし、またソフトウェアが誰か他人によって改変され、それが次々と頒布されていったとしても、その受領者は彼らが手に入れたソフトウェアがオリジナルのバージョンでは無いこと、そして原作者の名声は他人によって持ち込まれた可能性のある問題によって影響されることがないということを周知させたいと思います。

最後に、ソフトウェア特許がいかなるフリーのプログラムの存在にも不断の脅威を投げかけていますが、私たちは、フリーなプログラムの再頒布者が個々に特許ライセンスを取得することによって、事実上プログラムを独占的にしてしまうという危険を避けたいと思います。こういった事態を予防するため、私たちはいかなる特許も誰もが自由に利用できるようライセンスされるか、全くライセンスされないかのどちらかでなければならないことを明確にしました。 
\end{quote}


\end{document}
